\documentclass{ctexart}
\usepackage{xltxtra}
\usepackage{url}
\usepackage{amsmath}
\usepackage{graphicx}

\title{The Mandelbrot Set的生成和探索}
\author{吴声炜\\专业:数学与应用数学\\学号:3210102945}
\begin{document}
\include{abstract_cn}
\include{abstract_en}
\include{contents}
  

\begin{abstract}
  \verb|The Mandelbrot Set|是由一个迭代公式在复平面上构成的点集,并借由计算机可以绘制出它的图象,我们可以发现它的图象具有典型的分形结构,并且通过改变图象的参数可以得到十分有趣的例子。
\end{abstract}



\section{引言}

本论文将介绍\verb{The Mandelbrot`Set}的发现以及相关的数学理论,绘制\verb{The Mandelbrot`Set}图象所需的算法,介绍图象所具有的特点并绘制一些有趣的图象。

\section{The Mandelbrot Set的背景}

\verb|The Mandelbrot Set|是一个点集。这个点集均出自公式$z_{n+1}=z^2_n+c$,对于非线性迭代公式$z_{n+1}=z^2_n+c$,所有使得无限迭代后的结果保持有限数值的复数$z$的集合连通的$c$,构成s\verb|The Mandelbrot Set|。



\section{The Mandelbrot Set的数学理论}



\section{绘制The Mandelbrot Set图象的算法}


\section{图象例子}


\section{结论}


\bibliographystyle{plain}
\bibliography{bibfile}


\end{document}
